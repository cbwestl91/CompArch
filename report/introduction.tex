\IEEEPARstart{O}{ne} of the core problems in computer performance development is the growing processor - memory gap. While the processor speed has been increasing rapidly through the past 30 years, the memory access speed has been lagging behind. The major issue with memory access speed is related to the principle of locality. In order to have a lot of memory, you have to space it out, and traversing this space takes time. The prefetcher is an attempt to minimize the gap by fetching memory from RAM to cache, memory on the chip, but because of space requirements and limits to complexity, the prefetchers are still limited and subject to a lot of different algorithms and optimizations.

The goal of this project is to make a prefetcher that increases runtime performance by determining what memory block will be accessed before it's needed. Based on the DCPT algorithm described in \cite{reference:jahre}, an attempt to increase performance on a collection of benchmarks running on a modified M5 hardware simulator. 


