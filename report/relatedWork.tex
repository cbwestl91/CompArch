The most basic prefetcher is called a sequential prefetcher. When a CPU is requesting a memory adress, the prefetcher will also fetch the memory location. The performance of this prefetcher is very limtited since it does not have any form of heuristic. An improvement of this prefetcher is called tagged sequential prefetcher, which divides the sequences into blocks.

What these have in common, is that the prefetching is not based on statistics. They are instead deterministic fetches where a fixed group or single memory adress is fetched. To improve the performance, prefetching based on statistics were developed. These prefetchers will predict the next fetch statistically.   

A Reference Prediction Tables RPT is a prefetcher that is based on statistics. This type of prefetcher was first proposed in 1995 by Chan and Baer. A RPT contains a table with addresses of the missed fetches as index. Another variant of this is PC/DC prefetching. 

An another type of prefetching is called Delta Correlating Prediction, as described in \cite{reference:jahre}. Their purpose was to combine the delta correlating design of PC/DC and the storage efficiency of RPT. Compared to RC/DC, the DCPT reduces the complexity of the prefetching. This is done by recomputing the delta.
